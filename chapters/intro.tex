\chapter{Introduction}
\label{ch:intro}

\section{Motivation}
The COVID-19 pandemic has dramatically transformed the landscape of education, necessitating the adoption of distance learning as the new norm. As a result, students around the world faced the challenges of navigating online education without the traditional support system of physical classrooms and study groups. Recognizing the need for a solution to facilitate effective remote learning, the concept of Study Buddy was conceived.

Study Buddy, a mobile application developed as part of this Bachelor's thesis, aims to address the difficulties encountered by students in finding study groups and study partners in the digital learning environment. The idea originated during a business fundamentals class in 2020, where the task was to create an application, that would assist students during the pandemic.

Working collaboratively with another student from the same business fundamentals group, we envisioned Study Buddy as a platform, where students could connect with others studying similar courses or pursuing similar majors within their respective universities. By creating a profile that includes university, language, major, and selected courses, students can easily find study buddies who share common academic interests.

The underlying motivation for developing Study Buddy stemmed from the realization that study buddies play a pivotal role in the learning process, fostering collaboration, knowledge exchange, and academic support. However, amidst the challenges of online education, finding suitable study partners became increasingly difficult for our cohort, which commenced their degree program during the pandemic.

The motivation to pursue Study Buddy as a Bachelor's thesis project further intensified after achieving recognition in an online UI hackathon, where the concept garnered third place. This acknowledgment bolstered our belief in the application's potential impact on enhancing the learning experience for students.

By formalizing the concept of Study Buddy within this thesis documentation, we aspire to delve deeper into the development process, including the design, implementation, and evaluation of the application. Through this endeavor, we aim to contribute to the broader discourse on digital education and empower students to establish meaningful connections and collaborative learning environments in the context of distance learning.

In summary, this thesis aims to explore the development of Study Buddy, a mobile application designed to alleviate the challenges of online education by facilitating the formation of study groups and partnerships. By fostering collaboration and knowledge-sharing, Study Buddy endeavors to enhance the overall learning experience for students navigating the world of digital education.
\section{Collaboration}
The Study Buddy application was developed as a collaborative effort between myself and my colleague, Timilehin Bisola-Ojo. As part of a team of two, we divided the responsibilities based on our individual skills and expertise to deliver a comprehensive solution for the challenges posed by remote learning.

In this two-pronged approach, my role primarily revolved around frontend development. I focused on building a user-centric interface, ensuring the application was intuitive and engaging for students navigating the digital learning environment.

Timilehin, on the other hand, was responsible for backend development. His tasks involved constructing the underlying systems and structures that ensured the app's performance and scalability. In addition to this, he developed an admin website, providing a crucial tool for managing the application and monitoring its usage.

This clear division of roles enabled us to effectively progress in our development process, with each contributing unique strengths to the project. The result of our collaboration is Study Buddy - a practical and user-friendly application designed to address the unique challenges of distance learning.
\section{Thesis Structure}
The thesis consists of 4 main chapters including introduction. In addition to these chapters, the thesis includes a bibliography, which lists all the relevant sources and references used throughout the project development, a list of figures, a list of codes and an appendix  highlighting the work on backend development and development of the web app used for this application's administration, in a separate documentation.

Chapter 2: User Documentation \newline
This chapter covers the installation process and provides step-by-step instructions on how to install the app. The aim of this chapter is to ensure that users can easily navigate and utilize the app.

Chapter 3: Developer Documentation \newline
This chapter offers an in-depth exploration of the implementation structure.It provides developers with detailed insight into the architecture and design choices made during the development process, enabling a deeper understanding of the project's underlying technology.

Chapter 4: Conclusion \newline
The concluding chapter serves as the summary of the entire project. It presents an overview of the project's future direction, identifying potential areas for further exploration and improvement.

By following this structure, the thesis aims to provide a comprehensive understanding of the application, its development process, and its potential for future growth and enhancement.